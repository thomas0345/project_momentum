\documentclass[a4paper,12pt,twoside]{article}

\usepackage{amsmath}
%\usepackage{cite}
\usepackage[hidelinks]{hyperref}
\usepackage[includeheadfoot,margin=30mm]{geometry}
\usepackage{subcaption}
\usepackage{graphicx}
\usepackage{booktabs}
\usepackage{multirow}
\usepackage{siunitx}
\usepackage{natbib}
\usepackage{url}
\usepackage[english]{babel}
\usepackage{blindtext}

\usepackage{fancyhdr}
\pagestyle{fancy}
\fancyhf{}
\fancyhead[LE,RO]{\leftmark}
\fancyfoot[CE,CO]{\thepage}


\geometry{left=30mm,right=30mm,top=35mm,bottom=30mm,headheight=15pt}

\graphicspath{{./img/}}


\begin{document}
	
	\begin{titlepage}
		\centering
		\begin{figure}[!h]
			\centering
			\includegraphics[width=0.5\textwidth]{figures/UZH.png}
		\end{figure}
		\Large{\textbf{Optimal Lookback Window in Time-Series Momentum}\\}

		
		\vfill
		
		\large{\href{mailto:thomas.meier6@uzh.ch}{Thomas Meier} (19-738-400) \\
				\href{mailto:thomas.meier6@uzh.ch}{Ana Jazbinsek} (22-737-753)\\}
		
		\vfill
			
		\large{\today}
		
		\vfill
		\vfill
	
		\large{Department of Banking \& Finance \\ Digital Tools for Finance \\ Dr. Igor Pozdeev\\ }
	
		\vfill
		\begin{abstract}
			Abstract.  This paper investigates the effectiveness of momentum-based strategies in the context of country exchange-traded funds (ETFs), offering a detailed exploration of their performance under varying lookback periods and transaction cost considerations. The study examines a universe of global country ETFs, applying a long-only strategy, with weights determined based on past returns over different lookback windows. The analysis confirms that shorter lookback periods (up to 12 months) outperform longer ones, aligning with existing literature that suggests diminishing predictive power for extended lookback windows. The results demonstrate that momentum strategies outperform passive benchmarks, even after accounting for transaction costs, underlining their robustness in realistic market scenarios. Strategies utilizing optimal lookback windows achieve superior Sharpe ratios, validating the alpha-generating potential of momentum. However, poorly chosen lookback periods can lead to underperformance compared to the benchmark, emphasizing the importance of parameter selection. 
			\vspace{3mm}
			
			\textbf{Keywords:}  Momentum, Lookback, Investment Strategy
		\end{abstract}
		
		
	\end{titlepage}

    \clearpage

	\pagenumbering{roman}
    \setcounter{page}{1}
	\tableofcontents
	
	\clearpage
	\listoffigures
	
	\listoftables
	
	\clearpage
    \pagenumbering{arabic}
    \setcounter{page}{1}
    \setlength{\parindent}{0pt}
    \setlength{\parskip}{8pt}
%---------------------------------------

\section{Introduction}
Cross-sectional momentum strategies, which involve ranking assets based on past performance and constructing portfolios that long top performers while shorting underperformers, have been extensively studied in financial literature. The seminal work by \cite{jegadeesh1993returns} demonstrated that such strategies yield significant positive returns in equity markets. Subsequent research has confirmed the robustness of the momentum effect across various asset classes and geographies, challenging the efficient market hypothesis postulated by \cite{fama1970efficient}.
\\~\\
A critical aspect of implementing such momentum strategies is determining the optimal lookback period, e.g. the historical timeframe used to assess past performance. The choice of lookback window significantly influences the strategy's effectiveness, as it affects the identification of genuine performance trends versus short-term anomalies. Studies have explored various lookback periods, ranging from short-term (e.g., 1-3 months) to long-term (e.g., 12 months), to ascertain their impact on returns. While \cite{jegadeesh1993returns} found significant outperformance for momentum strategies between horizons from three to twelve months, they also found evidence of a reversal of this trend at longer horizons in the following years. They attribute this reversal effect to overreaction hypotheses similar to bubble dynamics but did not investigate it much further.
Other studies such as \cite{nagel2001overreaction}, investigated specifically this long-term reversal effect of momentum and found that this reversal effect can largely be attributed to predictable book-to-market effects.\footnote{Past winners that outperform significantly decrease their relative B/M ratio compared to peers.} After controlling for this effect, \cite{nagel2001overreaction} finds that momentum seems to persist further.
\\~\\
Given this background, this project aims to examine different lookback periods for a cross-sectional momentum strategy in equity markets. Instead of the traditional 12-1 month momentum portfolios we examine lookback windows up to 24 months. We ultimately implement a momentum strategy that goes long top three performers and doesn't short, however the user can adapt the number of lookback periods, number of long legs, short legs and transaction costs in the code provided.
\\~\\
Understanding the implications of different lookback windows is essential for both academics and practitioners aiming to optimize momentum-based investment strategies. This paper contributes to the existing literature by systematically analyzing the performance of cross-sectional momentum strategies across various lookback periods in equity markets, providing insights into their relative effectiveness and guiding the selection of appropriate parameters for portfolio construction.
\newpage
\section{Literature Review}
Momentum investing, a strategy that capitalizes on the persistence of asset price trends, has been a focal point in financial research for decades. The foundational study by \cite{jegadeesh1993returns} unveiled that stocks exhibiting superior returns over a 3- to 12-month period tend to continue outperforming in subsequent months, challenging the efficient market hypothesis. This phenomenon, termed the "momentum effect," has been observed across various asset classes and markets prompting extensive academic inquiry \citep{asness2013value, baltas2013momentum}.

%%%%%%
Over time, researchers have proposed multiple explanations for the momentum effect. Behavioral theories suggest that investor biases, such as delayed overreaction to information, contribute to price continuations. Conversely, risk-based explanations posit that momentum profits are compensation for bearing systematic risks not captured by traditional asset pricing models. Despite these efforts, a consensus on the underlying causes of momentum remains elusive \citep{barberis1998model, grinblatt2005prospect, bikhchandani1992theory}.

Momentum strategies are primarily categorized into two types: cross-sectional and time-series momentum. Cross-sectional momentum, as examined by \cite{jegadeesh1993returns}, involves ranking assets based on relative past performance, going long on top performers and short on underperformers. In contrast, time-series momentum, explored by \cite{moskowitz2012time}, assesses each asset's own past returns, taking positions based on whether past performance was positive or negative. While both strategies aim to exploit trend persistence, they differ in implementation and underlying assumptions.

This paper focuses on cross-sectional momentum strategies, specifically analyzing the impact of varying lookback periods on performance. Prior studies have investigated the influence of different formation and holding periods on momentum returns. For instance, \cite{jegadeesh1993returns} found that strategies with formation periods of 3 to 12 months yielded significant positive returns. However, \cite{nagel2001overreaction} provided some evidence for momentum at even longer horizons. Moreover, the optimal look back window may vary across markets, which necessitates further examination.
\newpage
\section{Dataset}
The project makes use of the Yahoo Finance API, in order to download the desired time frame for the analysis. Table~\ref{tab:country_etfs} illustrates the seventeen countries included in the analysis. Specifically, the selected sample includes nine developed markets and six emerging markets.
The analysis relies on MSCI as the index provider and Blackrock as the Exchange-Traded Fund (ETF) provider. Given, the analysis works with ETFs directly instead of indices, product costs are already included in the total return of the assets. Note that the Total Expense Ratio (TER) quantifies the annual operating costs of an ETF, encompassing management fees, administrative expenses, and other operational charges. Table~\ref{tab:country_etfs} outlines these costs in the TER column. The average TER of the ETFs in the sample is 0.529\%.
However, additional expenses such as transaction costs and brokerage commissions are not included in the TER and hence are modeled separately, as outlined in section~\ref{sec:methodology}.
\begin{table}[ht]
    \centering
    \begin{tabular}{>{\raggedright}p{3.5cm} >{\raggedright}p{7cm} >{\raggedright}p{1.5cm} >{\raggedright\arraybackslash}p{1.5cm}}
        \toprule
        \textbf{Country} & \textbf{ETF Name} & \textbf{Ticker} & \textbf{TER} \\
        \midrule
        USA             & iShares MSCI USA ETF             & EUSA   & 0.09\% \\
        China           & iShares MSCI China ETF           & MCHI   & 0.59\% \\
        Japan           & iShares MSCI Japan ETF           & EWJ    & 0.50\% \\
        India           & iShares MSCI India ETF           & INDA   & 0.64\% \\
        Brazil          & iShares MSCI Brazil ETF          & EWZ    & 0.59\% \\
        Canada          & iShares MSCI Canada ETF          & EWC    & 0.50\% \\
        Mexico          & iShares MSCI Mexico ETF          & EWW    & 0.50\% \\
        South Korea     & iShares MSCI South Korea ETF     & EWY    & 0.59\% \\
        Germany         & iShares MSCI Germany ETF         & EWG    & 0.50\% \\
        United Kingdom  & iShares MSCI United Kingdom ETF  & EWU    & 0.50\% \\
        Australia       & iShares MSCI Australia ETF       & EWA    & 0.50\% \\
        Switzerland     & iShares MSCI Switzerland ETF     & EWL    & 0.50\% \\
        Hong Kong       & iShares MSCI Hong Kong ETF       & EWH    & 0.50\% \\
        Singapore       & iShares MSCI Singapore ETF       & EWS    & 0.50\% \\
        Taiwan          & iShares MSCI Taiwan ETF          & EWT    & 0.59\% \\
        Italy           & iShares MSCI Italy ETF           & EWI    & 0.50\% \\
        Spain           & iShares MSCI Spain ETF           & EWP    & 0.50\% \\
        \bottomrule
    \end{tabular}
    \caption[Country Universe ETF Tickers]{Countries included in the universe with their respective MSCI ETFs, Yahoo Finance tickers, and Total Expense Ratios (TER)}
    \label{tab:country_etfs}
\end{table}

As the first step of the analysis we pull daily ETF price data from Yahoo finance for the period between 01-01-2014 to 20-09-2024.  To better visualize the price data of ETFs we show the descriptive statistics in Table~\ref{tab:summary_stats_price}.

\begin{table}[h!]
\centering
\begin{tabular}{lrrrrrrr}
\hline
\textbf{} & \textbf{Mean} & \textbf{Std} & \textbf{Min} & \textbf{25\%} & \textbf{50\%} & \textbf{75\%} & \textbf{Max} \\
\hline
USA         &  56.74 &  17.60 & 31.13 & 39.85 & 52.52 & 73.64 & 95.01 \\
China       &  18.09 &   3.37 & 10.71 & 15.61 & 17.29 & 21.29 & 26.82 \\
Japan       &  27.26 &   5.78 & 15.56 & 22.92 & 25.20 & 33.09 & 41.15 \\
India       &  25.05 &   3.45 & 15.84 & 22.51 & 24.62 & 27.39 & 33.09 \\
Brazil      &  18.51 &   2.74 & 12.83 & 16.08 & 18.73 & 20.52 & 25.03 \\
Canada      &  24.44 &   4.82 & 15.45 & 21.40 & 23.61 & 27.08 & 38.85 \\
Mexico      &  51.88 &   9.00 & 35.85 & 44.17 & 51.67 & 58.07 & 71.97 \\
South Korea &  34.67 &   8.04 & 22.94 & 27.33 & 31.91 & 42.15 & 52.85 \\
Germany     &  24.85 &   3.29 & 15.85 & 22.77 & 24.96 & 26.79 & 35.07 \\
UK          &  17.61 &   1.76 & 12.58 & 16.46 & 17.76 & 18.81 & 22.25 \\
Australia   &  29.97 &  11.10 & 14.66 & 20.12 & 25.30 & 40.24 & 56.88 \\
Switzerland &  27.00 &   3.34 & 16.69 & 24.85 & 26.79 & 28.99 & 37.61 \\
Hong Kong   &  45.54 &   8.64 & 22.66 & 40.04 & 44.19 & 49.72 & 69.99 \\
Singapore   &  58.22 &  10.93 & 36.46 & 50.10 & 56.94 & 63.97 & 90.72 \\
Taiwan      &  25.72 &   4.66 & 11.45 & 22.86 & 26.27 & 29.40 & 35.27 \\
Italy       &  34.04 &   8.76 & 19.15 & 26.96 & 31.43 & 41.42 & 58.13 \\
Spain       &  50.46 &  11.57 & 30.96 & 41.33 & 48.00 & 57.77 & 90.81 \\
\hline
\end{tabular}
\caption{Price statistics of the country ETFs}
\label{tab:summary_stats_price}
\end{table}

We can see that prices move around similar levels, with certain price indices being more volatile than others. However, even more representative is a look at the descriptive statistics of the return series of the ETFs. The return series is already aggregated to the monthly frequency, from the original daily price data. This is because we execute our strategy on a monthly basis thus using monthly returns as our momentum indicator. The descriptive statistics of the ETF monthly return series are shown in Table~\ref{tab:summary_stats_return}

\begin{table}[ht]
\centering
\begin{tabular}{lrrrrrrr}
\toprule
             & \textbf{Mean} & \textbf{Std} & \textbf{Min} & \textbf{25\%} & \textbf{50\%} & \textbf{75\%} & \textbf{Max} \\
\midrule
USA          & 0.91          & 4.91         & -21.40       & -1.56         & 1.03          & 3.60          & 19.82       \\
China        & 0.43          & 5.51         & -23.90       & -2.29         & 0.69          & 3.15          & 14.29       \\
Japan        & 0.50          & 5.02         & -22.44       & -2.13         & 0.67          & 3.42          & 16.79       \\
India        & 0.32          & 5.61         & -19.34       & -3.39         & 0.56          & 3.38          & 15.70       \\
Brazil       & 0.06          & 5.22         & -13.02       & -3.00         & 0.15          & 3.42          & 19.82       \\
Canada       & 0.50          & 6.04         & -23.81       & -3.21         & 0.59          & 4.35          & 22.47       \\
Mexico       & 0.38          & 3.92         & -9.56        & -1.64         & 0.48          & 2.66          & 10.78       \\
South Korea  & 0.51          & 4.04         & -8.63        & -2.22         & 0.56          & 3.62          & 9.86        \\
Germany      & 0.28          & 5.86         & -24.40       & -4.11         & 0.12          & 4.01          & 24.74       \\
UK           & 0.30          & 5.02         & -20.41       & -2.99         & 0.23          & 3.58          & 15.95       \\
Australia    & 0.82          & 5.24         & -14.87       & -2.42         & 0.81          & 4.14          & 20.71       \\
Switzerland  & 0.35          & 4.48         & -19.13       & -2.61         & 0.86          & 3.07          & 14.78       \\
Hong Kong    & 0.19          & 6.55         & -34.07       & -3.55         & 0.31          & 4.12          & 19.01       \\
Singapore    & 0.15          & 6.16         & -16.53       & -4.07         & 0.10          & 3.99          & 18.15       \\
Taiwan       & 0.54          & 9.26         & -40.19       & -5.04         & 0.30          & 7.01          & 22.38       \\
Italy        & 0.72          & 5.16         & -24.75       & -2.17         & 0.65          & 4.13          & 21.14       \\
Spain        & -0.04         & 6.43         & -16.87       & -3.70         & 0.40          & 3.69          & 26.64       \\
\bottomrule
\end{tabular}
\caption{Descriptive statistics of monthly returns of country ETFs in \%}
\label{tab:summary_stats_return}
\end{table}


In our strategy we also use the risk free rate, 3-month US Treasury rate, which we add or substract from the ETF return series in order to obtain total and excess returns. The descriptive statistics for this can be found in Table~\ref{tab:summary_stats_rf}


\begin{table}[ht]
\centering
\begin{tabular}{lccccccc}
\toprule
    & \textbf{Mean} & \textbf{Std} & \textbf{Min} & \textbf{25\%} & \textbf{50\%} & \textbf{75\%} & \textbf{Max} \\
\midrule
  Rf & 0.19 & 0.18 & 0.00 & 0.01 & 0.15 & 0.33 & 0.51 \\
\bottomrule
\end{tabular}
\caption{Risk free rate descriptive statistics in \% }
\label{tab:summary_stats_rf}
\end{table}


\newpage
\section{Methodology}\label{sec:methodology}
Momentum is based on the idea that a past trend will continue in the future. A past trend of good returns for a specific country ETF in our case should predict good returns of that country ETF in the future. Hence we first need to define how to define this past trend.
Thus, for a universe with $N$ ETFs, we define the momentum of an ETF $i$ as in equation~\ref{eq:mom}, where $r_{t,i}$ represents the total return of an ETF $i$ at time $t$ and $h$ represents the look back window over which we will iterate later on.
\begin{align}
MOM_{t,i} = r_{t-1,i}-r_{t-h,i}
 \label{eq:mom}
\end{align}
Given this definition, we can now sort all the ETFs in the universe at any given time and thereby create a ranking of the best performing all the way to the worst performing ETFs.\par
Next, we need to define the specific weights $w_{t,i}$ the strategy takes at each time step for each asset. This step depends on, if we choose to only work with the one leg (e.g. long only or short only) or with a long-short approach. For the long-only approach, equation~\ref{eq:Longonly} outlines the weights logic for going $k$ ETFs long.\footnote{For short-only one would just use the bottom k instead of the top k ETFs regarding the $MOM_{t,i}$ variable.}
Similarly equation~\ref{eq:LS} outlines the logic for the long-short portfolio, going $k$ long and $k$ short.\par
\begin{align}
w_{t,i} &=
\begin{cases}
\frac{1}{k}   & \text{if } MOM_{t,i} \text{ is among the top } k \text{ at time } t \\
0             & \text{otherwise}
\end{cases}
\label{eq:Longonly} \\[10pt]
w_{t,i} &=
\begin{cases}
\frac{1}{k}  & \text{if } MOM_{t,i} \text{ is among the top } k \text{ (long) at time } t \\
-\frac{1}{k} & \text{if } MOM_{t,i} \text{ is among the bottom } k \text{ (short) at time } t \\
0             & \text{otherwise}
\end{cases}
\label{eq:LS}
\end{align}

Having established the weight logic for the strategy, next we need to take into account the transaction costs.\footnote{In this paper we make use of proportional transaction costs, where we incur a cost $c$ for every currency unit. This cost is designed in order to include commissions, the bid-ask
spread, and any transaction taxes.}
We do this by calculating the turnover at each rebalancing period and multiply it with the proportional transaction cost, which then gets deducted from the monthly return.
For this, one needs to first take care of the return effect on the weights at each rebalancing period. This is needed, because between the rebalancing intervals, the ETFs all gain or lose some value and hence the actual portfolio weights differ from the desired weights of the strategy. Reestablishing the desired weights at each rebalancing needs trading and can be expressed in percentage of the portfolio, called turnover. Equation~\ref{eq:returnadj} takes care of the return adjustment. The idea is to compute the actual weights at the end of the period by using
\begin{align}
\hat{w}_{t,i} = w_{t-\epsilon,i} \frac{1+r_{t,i}}{1+r_{t,PF}}
 \label{eq:returnadj}
\end{align}
The total turnover per rebalancing period can then be calculated using equation~\ref{eq:turnover}.
\begin{align}
z_{t} = \sum_{j=1}^{K}\lvert w_{t,i} - \hat{w}_{t,i}\rvert
 \label{eq:turnover}
\end{align}
Given a proportional transaction cost $c$, one can now calculate the turnover cost at each rebalancing period by simply multiplying $z_t$ and $c$ and include this cost by deducting it from the monthly return of the strategy.\par
Using this approach one can easily replicate our calculations and verify the results. Note that a long-short portfolio built in a way described above is not using any capital. Hence, in order to compare it to long-only strategies or a benchmark, one needs to add the risk-free rate to the excess returns of the LS to get total returns and allow for a fair comparison of the strategies. However, the original strategy we run is a long-only strategy, thus in this case we do not add the risk-free rate because we already have total returns. We would deduct it if we wanted to view excess returns. The code of the project is written in a way that it understands the difference between a long-only and long-short strategy and can add the risk free rate in the right circumstances.


\newpage
\section{Empirical Results}

The code of the project is extensively parameterized in order for the user to be able to run different country ETF momentum strategies. However, in this paper we show our base strategy that is a long-only momentum strategy with the investment strategy as shown in Table~\ref{tab:country_etfs}. It takes 3 long legs - goes long the three best-performing country ETFs. It searches for best lookback periods in the range of 1 to 24 months. This means that when we check for the 16th lookback window we take return data from the entire 16-month lookback period (from 1 to 16 months back) not just the 16th previous month.  It also takes transaction costs of 10 basis points. All these parameters are adjustable in the code if the user so wishes.

In Table~\ref{table:performance1} and Table~\ref{table:performance2}, we present the main summary statistics for each of the lookback windows compared to the benchmark. We can see that in all but two cases momentum beats the becnhmark in terms of average total return over the period. In risk adjusted returns, benchmark beats nine of the lookback strategies, but still there are many momentum lookback strategies that perform much better. The strategies tend to be more volatile and have bigger extremes both in the positive and negative direction.

\begin{table}[ht]
\centering
\begin{tabular}{@{}c
                S[table-format=2.2]
                S[table-format=2.2]
                S[table-format=1.2]
                S[table-format=1.2]@{}}
\toprule
Lookback & {Avg Total Return (\%)} & {Avg XS Return(\%)} & {Std XS Return} & {Sharpe Ratio} \\
\midrule
1        & 9.06  & 6.22  & 0.21  & 0.29  \\
2        & 11.58 & 8.74  & 0.20  & 0.43  \\
3        & 9.28  & 6.43  & 0.20  & 0.32  \\
4        & 11.53 & 8.68  & 0.21  & 0.42  \\
5        & 12.45 & 9.60  & 0.19  & 0.50  \\
6        & 11.29 & 8.45  & 0.20  & 0.43  \\
7        & 11.98 & 9.13  & 0.20  & 0.45  \\
8        & 10.38 & 7.53  & 0.20  & 0.37  \\
9        & 12.50 & 9.66  & 0.21  & 0.46  \\
10       & 13.06 & 10.21 & 0.20  & 0.52  \\
11       & 10.92 & 8.07  & 0.21  & 0.39  \\
12       & 12.50 & 9.66  & 0.21  & 0.46  \\
13       & 8.73  & 5.89  & 0.22  & 0.27  \\
14       & 7.34  & 4.50  & 0.22  & 0.20  \\
15       & 10.14 & 7.30  & 0.20  & 0.37  \\
16       & 8.98  & 6.13  & 0.21  & 0.29  \\
17       & 11.60 & 8.75  & 0.20  & 0.44  \\
18       & 10.46 & 7.61  & 0.21  & 0.36  \\
19       & 9.54  & 6.69  & 0.21  & 0.32  \\
20       & 8.07  & 5.22  & 0.21  & 0.25  \\
21       & 10.28 & 7.43  & 0.20  & 0.37  \\
22       & 8.50  & 5.66  & 0.21  & 0.26  \\
23       & 8.27  & 5.42  & 0.21  & 0.26  \\
24       & 11.63 & 8.78  & 0.21  & 0.42  \\
\midrule
Benchmark & 8.16  & 5.31  & 0.16  & 0.33  \\
\bottomrule
\end{tabular}
\caption{Performance metrics for different Lookback periods}
\label{table:performance1}
\end{table}

\begin{table}[ht]
\centering
\begin{tabular}{@{}l
                c
                c
                c
                c@{}}
\toprule
Lookback & Worst (\%) & Best (\%)   & Skewness & Kurtosis \\
\midrule
1        & -32.72 & 22.39  & -0.86    & 8.07     \\
2        & -25.28 & 22.39  & -0.09    & 4.67     \\
3        & -24.00 & 22.39  & 0.07     & 4.18     \\
4        & -27.39 & 22.39  & -0.26    & 5.63     \\
5        & -24.00 & 22.39  & 0.09     & 5.28     \\
6        & -24.00 & 22.39  & -0.05    & 4.49     \\
7        & -25.49 & 22.39  & -0.08    & 4.46     \\
8        & -24.33 & 22.39  & 0.23     & 4.44     \\
9        & -25.78 & 22.39  & 0.10     & 4.60     \\
10       & -16.44 & 22.04  & 0.73     & 2.41     \\
11       & -23.04 & 22.39  & 0.19     & 3.76     \\
12       & -23.04 & 22.04  & 0.16     & 3.23     \\
13       & -30.26 & 22.04  & -0.48    & 6.13     \\
14       & -30.26 & 22.04  & -0.39    & 5.52     \\
15       & -21.54 & 21.03  & 0.21     & 3.10     \\
16       & -30.26 & 21.03  & -0.56    & 6.10     \\
17       & -19.50 & 22.04  & 0.37     & 2.85     \\
18       & -23.31 & 22.04  & 0.09     & 3.04     \\
19       & -23.31 & 22.04  & 0.09     & 3.02     \\
20       & -23.31 & 22.04  & 0.12     & 3.02     \\
21       & -14.98 & 22.04  & 0.59     & 1.83     \\
22       & -23.67 & 22.04  & 0.09     & 2.90     \\
23       & -23.31 & 21.03  & 0.07     & 2.69     \\
24       & -20.09 & 22.04  & 0.29     & 2.38     \\
\midrule
Benchmark & -20.19 & 13.71  & -0.51    & 3.12     \\
\bottomrule
\end{tabular}
\caption{Best and worst monthly returns together with overall skew and kurtosis for different lookback periods}
\label{table:performance2}
\end{table}


\clearpage

To visualize the performance outlined in Table~\ref{table:performance1} and Table~\ref{table:performance2}, we plot the graph shown in Figure~\ref{fig:bm}. It displays the best two and worst two performing momentum lookback strategies over our sample period for our parameter selection (3 long legs and no short legs). The best and worst performers are differentiated with color, the best being in blue and the worst being in orange. The top lookback performers outperform during the majority of the backtest period and have a clearly higher total return NAV by mid 2024. We also show the benchmark in gray, to see whether there are any strategies that don't even beat the benchmark.  As expected the different momentum strategies consistently outperform the equally weighted passive benchmark approach, with a slight exception of the 1 month lookback strategy that at times has a lower total return NAV than the benchmark. The graph also shows well how the shorter term (up to 12 months) lookbacks perform better than the long term (more than 12 months) lookbacks. This confirms what we've already mentioned in the literature review, that the momentum signal gets weaker if the lookback window is too big.

\begin{figure}[h]
    \centering
    \includegraphics[width=15cm]{figures/fig_bm.pdf}
    \caption{Comparison of the two best and two worst performing lookback strategies and the benchmark without transactions costs}
    \label{fig:bm}
\end{figure}

\clearpage

In Figure~\ref{fig:costs} we display the comparison of strategy and benchmark performance with and without trading costs. We plot this graph to check whether our strategy out-performance still holds even after accounting for the transaction costs. This makes the comparison more realistic for real-life application. As expected benchmark transaction costs are lower than the strategy transaction costs - the difference between lines representing strategy without and with costs are more wide apart then benchmark and benchmark with costs. This is due to a higher turnover volume for the individual strategies compared to the benchmark. We actively change the weights in the strategies, whereas we only rebalance back to equally weighted in the benchmark. Also, even after accounting for costs the top performing strategies out-perform the benchmark. This proves that even in a more realistic setting the momentum strategy generates alpha.

\begin{figure}[h]
    \centering
    \includegraphics[width=15cm]{figures/fig_costs.pdf}
    \caption{Comparison of best performing lookback strategies and the benchmark both with and without transaction costs}
    \label{fig:costs}
\end{figure}



\clearpage
\newpage
\section{Conclusion}

The analysis and implementation of momentum-based strategies across country ETFs provides several important insights. First, the study confirms that shorter lookback periods, up to 12 months, generally outperform longer lookback periods. This observation aligns with existing literature, which indicates that the predictive power of momentum signals diminishes as the lookback window becomes excessively extended.

Moreover, the results show that best momentum strategies outperform the benchmark passive strategy, even after accounting for transaction costs. This demonstrates the robustness of these strategies in more realistic market scenarios. The superior Sharpe ratios observed for the best-performing lookback windows, compared to the benchmark, further highlight the effectiveness of momentum in generating alpha. However, the worst performing lookback strategies can also perform worse than the passive benchmark. Thus, the choice of a lookback period is essential in generating meaningful returns. 

Although the strategies incur higher turnover and associated costs than the benchmark, they still deliver substantial risk-adjusted returns. This underscores the importance of carefully managing transaction costs while leveraging momentum signals. 

Overall, this study validates momentum as a reliable and effective strategy for country ETFs, emphasizing the critical role of lookback window selection and cost considerations in optimizing performance.

\clearpage
\bibliographystyle{plainnat}
\bibliography{references}
	
	
\end{document}
